Seen as microfactories, microalgaes are organisms already in use in many industries. Howewer, the molecules of interests they're making are enclosed and so far the only way of releasing is done by killing the biomass, filtering and extracting the molecule. Working on membrane porosity is challenging and could help finding a "clean way" of producing such molecules. Controlled electroporation could be a way. Previous experiments shows that there is a voltage dependency to form non permanent holes in the membrane.  This reversible damage will allow the escape of some molecules and the repair by the microalgae to take place. Howewer, measuring the effects on electroparation different parameters is challenging. 
